\subsection{}
\subsubsection{}
To start simple, lets confine ourselves first to the $\mathbb{R}^2$ case, where for the unit circle $n=1$. We then have a mapping
\[
\sigma : \mathbb{S}^1 \backslash \{N\} \rightarrow \mathbb{R}^2. 
\]
If we parameterize this around a point $p$ on $\mathbb{S}^2$ in the upper-half unit circle for $(x_p^1,x_p^2)\in \mathbb{R}^2$:
\[
x_p^2 = \sqrt{1-(x_p^1)^2}
\]
for which one can span a line $L$ through $N$ and $(x_p^1,x_p^2)$, namely:
\[
	L = 1 + ax^1 = 1 + \dfrac{\sqrt{1-(x_p^1)^2}-1}{x_p^1}x^1
\]
such that when $\Gamma(L)=(u, 0)$, it crosses the horizontal line $(x_1, 0) \in \mathbb{R}^2$:
\[
\begin{split}
	\dfrac{\sqrt{1-(x_p^1)^2}-1}{x_p^1}x^1 &= -1 \\
	x_1 &= \dfrac{x_p^1}{1-x_p^2}
\end{split}
\]
Thus, the projection $\sigma$ maps the intersection $N$ and any point $p = (x_p^1, x_p^2)$, such that one can write: 
\[
\sigma(x^1, x^2) \backslash \{N\} \rightarrow \left(\dfrac{x^1}{1-x^2}, 0\right).
\]
One can notice that the half-unit circle is not necessary to be parametrized for $x^2$ in terms of $x^1$, the square root term can be entirely replaced by $x^2$ itself. This  implies that not just $\mathbb{S}^1\backslash \{N\} $, but basically any point in $\mathbb{R}^2\backslash \{N\} $ can be mapped to the lower dimensional projective hyperplane.\footnote{This does come at a cost of uniqueness, points on the same line spanning through $N$ will of course end up in the same location in the hyperplane $\mathbb{R}^n$.}

The 2-dimensional example can be easily extended to the $n$-dimensional form. The line spanning through $x_p$ and $N$ then has a differential form defined as:
\[
L = x_p^{n+1} = 1 + \dfrac{x_p^{n+1} - 1}{x_p^1} x^1 + \dfrac{x_p^{n+1} - 1}{x_p^2} x^2 + \dots + \dfrac{x_p^{n+1} - 1}{x_p^{n}} x^{n}
\]
i.e., every coordinate $x^i$ instead has a mapping 
\[
\dfrac{x^i}{1-x^{n+1}} \quad \forall i \in 1\dots n
\]

\subsubsection{}
For $\sigma^{-1}(u)$, we take a single element:
\[
	\dfrac{2u^i}{||u||^2 + 1} \quad \textnormal{where} \quad u^i = \dfrac{x^i}{1-x^{n+1}}
\]
such that
\[
	\dfrac{2u^i}{||u||^2 + 1} = \dfrac{2x^i}{(1-x^{n+1})(||u||^2 + 1)}
\]
Here, we can rewrite $||u||^2$ as
\[
||u||^2 = \sum_{i\in 1\dots n}\dfrac{(x^i)^2}{(1-x^{n+1})^2} = \dfrac{\sum_{i\in 1\dots n}(x^i)^2}{(1-x^{n+1})^2}
\]
so that we are able to write 
\[
\begin{split}
\dfrac{2x^i}{(1-x^{n+1})(||u||^2 + 1)} &= \dfrac{2x^i}{(1-x^{n+1})\left(\dfrac{\sum_{i\in 1\dots n}(x^i)^2}{(1-x^{n+1})^2} + 1\right)} \\
&= \dfrac{2x^i}{\dfrac{\sum_{i\in 1\dots n}(x^i)^2}{1-x^{n+1}} + 1-x^{n+1}}\\
&= \dfrac{2x^i}{\dfrac{\sum_{i\in 1\dots n}(x^i)^2 + (x^{n+1})^2 -2x^{n+1} + 1}{1-x^{n+1}}}
\end{split} 
\]
where for $x\in \mathbb{S}^n$
\[
\sum_{i\in 1\dots n}(x^i)^2 + (x^{n+1})^2 = ||x||^2 = 1
\]
such that
\[
\begin{split}
	\dfrac{2x^i}{\dfrac{\sum_{i\in 1\dots n}(x^i)^2 + (x^{n+1})^2 -2x^{n+1} + 1}{1-x^{n+1}}} = \dfrac{2x^i}{\dfrac{1 - 2x^{n+1} + 1}{1-x^{n+1}}} = \dfrac{2x^i}{\dfrac{2(1 - x^{n+1})}{1-x^{n+1}}} = x^i
\end{split} 
\]
Concluding, $\sigma^{-1}(u)$ maps all $u^i$ to $x^i$.

\subsubsection{}
A commutative diagram of the operation can be seen in Fig. \ref{tikz:composite}. To support this operation, the intersection of $ \mathbb{S}^n \backslash \{N\}$ and $ \mathbb{S}^n \backslash \{S\}$ is to be used in the transition map (where the charts are compatible). These charts are thus part of an atlas $\mathcal{A}$ that encompasses them both.
\[
 \mathbb{S}^n \backslash \{N\} \cap  \mathbb{S}^n \backslash \{S\} =  \mathbb{S}^n \backslash \{N, S\}
\]
The charts are compatible, such that their union within the Atlas defined as:
\[
	 \mathbb{S}^n \backslash \{N\} \cup  \mathbb{S}^n \backslash \{S\} =  \mathbb{S}^n
\]
covers $\mathbb{S}^n$.

\begin{figure}[!ht]
	\centering
	\begin{tikzcd}
		{} & x \in \mathbb{S}^n \backslash \{N\} \arrow[d, "\sim"]    \subset \mathbb{R}^{n+1}       & u \in \mathbb{R}^n \arrow[l, "\sigma^{-1}"] \\
		& x \in \mathbb{S}^n \backslash \{S\} \arrow[r, "\tilde{\sigma}"]\subset \mathbb{R}^{n+1}  & \tilde{u} \in \mathbb{R}^n              
	\end{tikzcd}
\caption{Commutative diagram for $\tilde{\sigma}\circ\sigma^{-1}$}
\label{tikz:composite}   
\end{figure}
Write the standard inverse map for $\sigma$:
\begin{equation}
	\label{eq:invsigma}
\sigma^{-1}(u) = \left(\dfrac{2u^1}{||u||^2 + 1}, \dots ,\dfrac{2u^n}{||u||^2 + 1}, \dfrac{||u||^2-1}{||u||^2+1} \right)
\end{equation}
and its forward south-pole equivalent:
\begin{equation}
	\label{eq:tildesigma}
	\tilde{\sigma}(x) = -\sigma(-x) = \left(\dfrac{-x^1}{1+x^{n+1}}, \dots , \dfrac{-x^n}{1+x^{n+1}}\right),
\end{equation}
we can force the elements of \eqref{eq:invsigma} into \eqref{eq:tildesigma} such that we end up with:
\[
\tilde{\sigma}\circ\sigma^{-1} = \left(\dfrac{\dfrac{-2u^1}{||u||^2 + 1}}{1 + \dfrac{||u||^2 - 1}{||u||^2 + 1}}, \dots \right) = \left(\dfrac{-2u^1}{2||u||^2}, \dots \right) = \left( \dfrac{-u^1}{||u||^2}, \dots \right)
\]
i.e. 
\[
	\tilde{\sigma}\circ\sigma^{-1} : u \rightarrow \tilde{u} = \left( \dfrac{-u^1}{||u||^2} ,\dfrac{-u^2}{||u||^2}, \dots, \dfrac{-u^n}{||u||^2} \right)
\]
This map is bijective between $u$ and $\tilde(u)$ in $\mathbb{R}^n$ since every coordinate has a linear one-to-one correspondence, and is smooth if $||u|| \neq 0$.
To show that the atlas defines a smooth structure on $\mathbb{S^n}$,  we are left with the task to show if the transition map 	$\tilde{\sigma}\circ\sigma^{-1} : u \cap \tilde{u} \rightarrow u \cap \tilde{u}$ is a smooth diffeomorhpsim.

\subsubsection{} 
The atlas of the stereographic projection basically performs the same operation with its charts as in example 1.2.14. The big difference is that we don't project straightly inward to a hyperplane, but at an angle with the line spanning the north/south pole and the hyperplane.With the explicit inclusion of both projections in the stereographic atlas, the north and south pole are included in the $\mathbb{S}^1$ structure as in ex. 1.2.14. Since both atlasses encompass the same manifold, they are related by an equivalence class and thus define a smooth structure.